% :Author: Refined Rubik's Cube with UC San Diego Colors
\documentclass[12pt]{standalone}
\usepackage{tikz}
\usepackage{verbatim}
\usepackage{fontspec}

\begin{comment}
:Title: Rubik's Cube with UC San Diego Colors and Refined Text
:Tags: 3D, Transformations, Text Enhancement

This design uses UC San Diego's official colors, enhances text readability, and maintains a professional aesthetic.
\end{comment}

\usetikzlibrary{positioning,shadings}
\begin{document}
\large
\pagestyle{empty}

% Define UC San Diego colors
\definecolor{ucsdblue}{RGB}{0, 51, 102}
\definecolor{ucsdyellow}{RGB}{255, 204, 0}
\definecolor{ucsdgray}{RGB}{166, 166, 166}
\definecolor{textcolor}{RGB}{255, 255, 255} % White text for contrast

\resizebox{5in}{!}{
\begin{tikzpicture}[
    every node/.style={minimum size=1cm, font=\bfseries\sffamily\large, align=center},
    cube face/.style={draw=black, very thick},
    text face/.style={font=\bfseries\sffamily\large, text=textcolor},
    shading/.style={left color=ucsdblue!80, right color=ucsdblue!50}
]

% Front face (divided into squares with UCSD blue)
\foreach \x in {0, 1, 2} {
    \foreach \y in {0, 1, 2} {
        \fill[ucsdblue] (\x,\y,0) rectangle ++(1,1);
        \draw[cube face] (\x,\y,0) rectangle ++(1,1);
    }
}

% Top face (divided into squares with UCSD yellow)
\foreach \x in {0, 1, 2} {
    \foreach \z in {0, 1, 2} {
        \fill[ucsdyellow] (\x,2,\z) -- ++(1,0) -- ++(0,-1) -- ++(-1,0) -- cycle;
        \draw[cube face] (\x,2,\z) -- ++(1,0) -- ++(0,-1) -- ++(-1,0) -- cycle;
    }
}

% Side face (divided into squares with UCSD gray)
\foreach \y in {0, 1, 2} {
    \foreach \z in {0, 1, 2} {
        \fill[ucsdgray] (2,\y,\z) -- ++(0,1) -- ++(0,0,-1) -- ++(0,-1) -- cycle;
        \draw[cube face] (2,\y,\z) -- ++(0,1) -- ++(0,0,-1) -- ++(0,-1) -- cycle;
    }
}

% Add "RGT Lab" text with enhanced readability and alignment
\node[text face] at (1.5, 1, 0) {RGT Lab}; % Front face
\node[text face, rotate=90] at (2, 1, 1) {RGT Lab}; % Side face
\node[text face] at (1, 2.5, 1) {RGT Lab}; % Top face

\end{tikzpicture}
}
\end{document}
